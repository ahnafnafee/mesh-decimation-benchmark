\documentclass[11pt, latterpaper]{article}

% --- PREAMBLE BLOCK (Modified for pdflatex compatibility) ---
\usepackage[margin=0.5in]{geometry} % Set 0.5-inch margins
\usepackage{fontenc}        % Use T1 font encoding (pdflatex compatible)
\usepackage[utf8]{inputenc}     % Allow UTF-8 input (pdflatex compatible)
\usepackage{amsmath}            % For math environments and symbols
\usepackage{hyperref}           % For clickable links
\usepackage{titlesec}           % To customize section headings
\titlespacing*{\section}{0pt}{1.5ex}{1ex}
\titlespacing*{\subsection}{0pt}{1ex}{0.5ex}

% --- CUSTOMIZE SECTION HEADINGS ---
% This makes unnumbered \section* headings \normalsize and bold
\titleformat{name=\section,numberless}{\normalsize\bfseries}{}{0em}{} 
\titleformat{name=\subsection,numberless}{\normalsize\bfseries}{}{0em}{}

% Remove page numbers for a proposal
\pagestyle{empty}

% Package for compact lists
\usepackage{enumitem}
\setlist{noitemsep, topsep=0.5ex, leftmargin=*}

% Reduce spacing between paragraphs slightly
\setlength{\parskip}{0.5\baselineskip}

\begin{document}

% --- TITLE SECTION ---
\begin{center}
    {\large \textbf{Performance Analysis of 3D Mesh Simplification Algorithms}} \\
    {\small \text{Ahnaf An Nafee (G00154707)}}
\end{center}
\vspace{-0.5em}

% --- PROBLEM STATEMENT ---
\section*{1. Problem Statement}
High-polygon 3D models create performance bottlenecks for real-time graphics applications. Mesh simplification algorithms are essential for reducing model complexity, but their performance is typically benchmarked on idealized data, which may not reflect real-world usage. This project addresses this gap by investigating the research question: \textit{Do the performance and geometric fidelity of common mesh simplification algorithms differ significantly when applied to clean, well-structured CAD models versus noisy, user-generated models?} The goal is to provide a rigorous, quantitative answer through formal experimental design and statistical analysis.

% --- METHODOLOGY (CONSOLIDATED SECTION) ---
\section*{2. Methodology}
A two-factor, between-subjects experimental design will be employed. The methodology consists of the experimental setup, data collection procedure, and the statistical analysis plan.

\subsection*{Experimental Design}
\begin{itemize}
    \item \textbf{Independent Variables (Factors)}:
    \begin{itemize}
        \item \textit{Factor 1: Simplification Algorithm} (2 Levels): Quadric Edge Collapse Decimation (QEM) and Clustering Decimation.
        \item \textit{Factor 2: Mesh Type} (2 Levels): ``Clean CAD'' models from the \textbf{ModelNet40} dataset and ``Organic/Scanned'' models from the \textbf{Thingi10K} dataset.
    \end{itemize}
    \item \textbf{Dependent Variables (Metrics)}:
    \begin{itemize}
        \item \textit{Geometric Fidelity}: Quantified by the \textbf{Two-Sided Hausdorff Distance} ($max(d(A,B), d(B,A))$).
        \item \textit{Computational Performance}: Measured by the \textbf{Wall-Clock Time} in seconds (averaged over 5 repetitions with warm-up).
    \end{itemize}
\end{itemize}

\subsection*{Procedure and Tools}
A preliminary data pre-processing step will ensure all models are correctly formatted. The main experiment will then be automated using a Python script that utilizes \textbf{MeshLab}'s Python library (\texttt{pymeshlab}) to apply each algorithm to a curated set of 30 models (15 from each dataset) and record the dependent variables. This ensures a reproducible and error-free data collection process.

\subsection*{Statistical Analysis Plan}
The collected data will be analyzed using a formal hypothesis-testing framework with a significance level of $\alpha = 0.05$.
\begin{itemize}
    \item \textbf{Hypotheses}: Two separate \textbf{Two-Way ANOVAs} will test the null hypotheses for the main effects of algorithm and mesh type, and for their interaction effect. The interaction effect ($H_{0,int}$) is of primary interest, as rejecting it would provide evidence that the optimal algorithm choice depends on the model's quality.
    \item \textbf{Analysis Pipeline}:
    \begin{enumerate}
        \item \textbf{Descriptive Statistics}: Calculate mean, standard deviation, standard error, and \textbf{95\% confidence intervals} for each of the 4 experimental groups.
        \item \textbf{Assumption Checks}: Validate the use of ANOVA by testing for normality (\textbf{Shapiro-Wilk test}) and homogeneity of variances (\textbf{Levene's test}).
        \item \textbf{ANOVA Execution}: Perform the Two-Way ANOVA using Python's \texttt{statsmodels} library to obtain F-statistics and $p$-values.
        \item \textbf{Post-Hoc Analysis}: If significant interaction effects are found, independent T-tests (Welch's) will be conducted to analyze simple main effects.
    \end{enumerate}
\end{itemize}

\end{document}