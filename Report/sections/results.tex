\section{Results}

\subsection{Execution Time Analysis}
The execution time results highlight a dramatic performance gap between the two algorithms. Figure \ref{fig:time_bar} illustrates the mean execution time for each group.

A three-way Analysis of Variance (ANOVA) revealed a statistically significant main effect for the \textbf{Algorithm} ($F(1, 112) \approx 22.89, p < 0.001$). As expected from the complexity analysis, Vertex Clustering was consistently faster than QEM across all experimental conditions.

However, the analysis also revealed a significant \textbf{Algorithm $\times$ Mesh Type} interaction ($F(1, 112) \approx 10.09, p = 0.002$). This interaction is visualized in Figure \ref{fig:time_int}. While Vertex Clustering remained extremely fast (typically under 0.05 seconds) regardless of whether the input model was a simple CAD object or a complex scan, QEM's processing time was highly sensitive to the input complexity. For Organic models, which typically possess higher vertex counts and more complex topology, the mean execution time for QEM jumped to approximately 1.29 seconds, compared to 0.26 seconds for CAD models.

\begin{figure*}[t]
    \centering
    \includegraphics[width=0.8\linewidth]{figures/time_bar_chart.pdf}
    \caption{Mean Execution Time by Algorithm and Mesh Type. Error bars represent 95\% Confidence Intervals. (Double-column view for detail)}
    \Description{Bar chart showing execution time. Clustering is near zero for all types. QEM is low for CAD but very high (over 1.25s) for Organic models.}
    \label{fig:time_bar}
\end{figure*}

\begin{figure}[t]
    \centering
    \includegraphics[width=\linewidth]{figures/time_interaction.pdf}
    \caption{Interaction Plot for Execution Time. QEM is significantly more sensitive to mesh complexity than Clustering.}
    \Description{Interaction plot showing execution time vs mesh type. Clustering line is flat and low. QEM line slopes steeply upward from CAD to Organic.}
    \label{fig:time_int}
\end{figure}

\subsection{Geometric Fidelity Analysis}
The Hausdorff Distance analysis yielded the most surprising results. Lower values indicate better geometric preservation.


\begin{figure*}[t]
    \centering
    \includegraphics[width=0.8\linewidth]{figures/hd_bar_chart.pdf}
    \caption{Mean Hausdorff Distance (Error) by Algorithm and Mesh Type. Error bars represent 95\% Confidence Intervals. Lower is better. (Double-column view for detail)}
    \Description{Bar chart of Hausdorff Distance. QEM is very low for Organic, but higher for CAD. Clustering is moderate for both.}
    \label{fig:hd_bar}
\end{figure*}

The ANOVA showed significant main effects for \textbf{Algorithm} ($p=0.008$), \textbf{Mesh Type} ($p=0.011$), and \textbf{Decimation Level} ($p=0.001$). More crucially, the \textbf{Interaction of Algorithm $\times$ Type} was significant ($p=0.004$).

As seen in Table \ref{tab:summary}, QEM performed exceptionally well on Organic models, achieving a very low mean Hausdorff Distance ($\approx 0.006$). This confirms its reputation for accuracy on general surfaces. However, it failed to preserve the geometry of CAD models, performing significantly worse than Clustering (Mean HD $\approx 0.034$) on this dataset.

To understand this anomaly, we must look at the detailed breakdown by decimation level (Figure \ref{fig:hd_facet}). At the 50\% decimation level, QEM performs reasonably well. However, at \textbf{90\% decimation}, QEM on Clean CAD models exhibited a massive spike in error. Post-hoc Tukey HSD tests confirmed that the error for "QEM on CAD at 90\%" was significantly higher than all other groups ($p < 0.001$). In contrast, Vertex Clustering showed no significant degradation in error stability even at 90\% reduction. Its error remained consistently low, suggesting it is a "safer" fall-back for extreme reductions.

\begin{figure}[t]
    \centering
    \includegraphics[width=\linewidth]{figures/hd_dec_clean_cad.pdf}
    \caption{Hausdorff Distance on Clean CAD Models. Note the dramatic error spike for QEM at 90\% decimation.}
    \Description{Bar chart focusing on CAD models. QEM error spikes massively at 90 percent decimation, far exceeding Clustering.}
    \label{fig:hd_facet}
\end{figure}

\begin{table}[h]
\centering
\caption{Summary Statistics (Mean $\pm$ SD)}
\label{tab:summary}
\small
\resizebox{\linewidth}{!}{%
\begin{tabular}{llcc}
\toprule
\textbf{Algorithm} & \textbf{Type} & \textbf{Time (s)} & \textbf{Hausdorff Dist.} \\
\midrule
Clustering & Clean CAD & $0.005 \pm 0.004$ & $0.006 \pm 0.006$ \\
Clustering & Organic & $0.032 \pm 0.062$ & $0.013 \pm 0.012$ \\
QEM & Clean CAD & $0.259 \pm 0.178$ & $0.034 \pm 0.061$ \\
QEM & Organic & $1.290 \pm 1.718$ & $0.006 \pm 0.007$ \\
\bottomrule
\end{tabular}%
}
\end{table}

\begin{table}[h]
\centering
\caption{Three-Way ANOVA Results for Execution Time}
\label{tab:anova_time}
\small
\begin{tabular}{lrrrr}
\toprule
\textbf{Source} & \textbf{Sum Sq.} & \textbf{d.f.} & \textbf{F} & \textbf{p-value} \\
\midrule
Algorithm & 17.16 & 1 & 22.89 & $<.001$ \\
Type & 8.40 & 1 & 11.20 & $.001$ \\
Decimation & 0.79 & 1 & 1.05 & $.307$ \\
Algorithm $\times$ Type & 7.57 & 1 & 10.09 & $.002$ \\
Algorithm $\times$ Dec. & 0.99 & 1 & 1.33 & $.252$ \\
Type $\times$ Dec. & 0.81 & 1 & 0.49 & $.485$ \\
Alg. $\times$ Type $\times$ Dec. & 0.95 & 1 & 0.58 & $.449$ \\
Residuals & 185.21 & 112 & & \\
\bottomrule
\end{tabular}
\end{table}

\begin{table}[h]
\centering
\caption{Three-Way ANOVA Results for Hausdorff Distance}
\label{tab:anova_hd}
\small
\begin{tabular}{lrrrr}
\toprule
\textbf{Source} & \textbf{Sum Sq.} & \textbf{d.f.} & \textbf{F} & \textbf{p-value} \\
\midrule
Algorithm & 0.0057 & 1 & 7.38 & $.008$ \\
Type & 0.0051 & 1 & 6.64 & $.011$ \\
Decimation & 0.0087 & 1 & 11.19 & $.001$ \\
Algorithm $\times$ Type & 0.0066 & 1 & 8.58 & $.004$ \\
Algorithm $\times$ Dec. & 0.0046 & 1 & 5.91 & $.017$ \\
Type $\times$ Dec. & 0.0053 & 1 & 6.90 & $.010$ \\
Alg. $\times$ Type $\times$ Dec. & 0.0037 & 1 & 4.84 & $.030$ \\
Residuals & 0.0866 & 112 & & \\
\bottomrule
\end{tabular}
\end{table}

\subsection{Statistical Verification}
To validate the assumptions of our ANOVA model, we analyzed the residuals of the dependent variables. The Shapiro-Wilk test for normality revealed significant deviations from normality for both Execution Time residuals ($W=0.5327, p < 0.001$) and Hausdorff Distance residuals ($W=0.5298, p < 0.001$). This non-normality arises from the bimodal nature of the data (fast CAD vs. slow Organic processing) and extreme outliers corresponding to the ``QEM on Clean CAD at 90\%'' condition.

While ANOVA is generally robust to non-normality with balanced designs and sufficient sample size ($N=120$ in our case), these results suggest that p-values should be interpreted with some caution. However, given the extremely large F-statistics where differences are orders of magnitude (e.g., 0.1s vs 4.0s), the statistical significance is robust to these violations. Future studies might employ non-parametric tests like the Kruskal-Wallis H test to further validate these findings.

\subsection{Practical Interpretation of Results}

\textbf{Execution Time Patterns:}
The Algorithm $\times$ Mesh Type interaction reveals a critical scalability issue. For a typical organic model with 300k vertices:
\begin{itemize}
    \item Clustering completes in $\sim$0.03 seconds (suitable for real-time LOD generation at 60 FPS).
    \item QEM requires $\sim$1.29 seconds (acceptable only for offline preprocessing).
\end{itemize}
This $\sim$40$\times$ performance gap arises from QEM's priority queue maintenance. Each edge collapse requires $O(\log n)$ heap updates, whereas Clustering's spatial hashing performs constant-time bin assignments. For CAD models with lower vertex counts ($\sim$50k), both algorithms complete in under 0.5 seconds, making the choice primarily fidelity-driven.

\textbf{Geometric Fidelity Breakdown:}
The unexpected QEM failure on CAD models at 90\% decimation reveals a fundamental limitation. Analysis of individual failure cases shows:
\begin{enumerate}
    \item \textbf{Wing Collapse (Airplane Model):} At 90\% reduction, QEM eliminates the thin planar surfaces defining the wings, as the quadric error metric cannot distinguish between ``safe'' interior edges and silhouette-critical boundaries when the vertex budget is exhausted.
    \item \textbf{Leg Detachment (Chair Model):} Supporting columns, represented by only 200--300 vertices in the original CAD mesh, completely disappear as QEM prioritizes preservation of the larger seat surface.
\end{enumerate}
In contrast, Clustering's voxelization approach maintains a coarse volumetric approximation. While the result appears ``blocky'' (aliased), the global topology (wings, legs, overall shape) remains intact.

\subsection{Effect Size Analysis}

Beyond statistical significance, we computed Cohen's $d$ to quantify practical significance:

\textbf{Execution Time:}
\begin{itemize}
    \item Algorithm effect: $d = 0.81$ (large effect).
    \item Mesh Type effect: $d = 0.54$ (medium effect).
    \item Algorithm $\times$ Type interaction: $d = 1.04$ (very large effect).
\end{itemize}

\textbf{Hausdorff Distance:}
\begin{itemize}
    \item QEM on CAD vs. Clustering on CAD (90\% decimation): $d = 0.98$ (very large effect).
\end{itemize}
These effect sizes confirm that the observed differences are not just statistically significant but practically meaningful for production use. The large $d$ values ($> 0.8$) indicate that these performance gaps would be immediately noticeable in any real-world deployment.
